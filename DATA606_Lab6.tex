% Options for packages loaded elsewhere
\PassOptionsToPackage{unicode}{hyperref}
\PassOptionsToPackage{hyphens}{url}
%
\documentclass[
]{article}
\usepackage{lmodern}
\usepackage{amssymb,amsmath}
\usepackage{ifxetex,ifluatex}
\ifnum 0\ifxetex 1\fi\ifluatex 1\fi=0 % if pdftex
  \usepackage[T1]{fontenc}
  \usepackage[utf8]{inputenc}
  \usepackage{textcomp} % provide euro and other symbols
\else % if luatex or xetex
  \usepackage{unicode-math}
  \defaultfontfeatures{Scale=MatchLowercase}
  \defaultfontfeatures[\rmfamily]{Ligatures=TeX,Scale=1}
\fi
% Use upquote if available, for straight quotes in verbatim environments
\IfFileExists{upquote.sty}{\usepackage{upquote}}{}
\IfFileExists{microtype.sty}{% use microtype if available
  \usepackage[]{microtype}
  \UseMicrotypeSet[protrusion]{basicmath} % disable protrusion for tt fonts
}{}
\makeatletter
\@ifundefined{KOMAClassName}{% if non-KOMA class
  \IfFileExists{parskip.sty}{%
    \usepackage{parskip}
  }{% else
    \setlength{\parindent}{0pt}
    \setlength{\parskip}{6pt plus 2pt minus 1pt}}
}{% if KOMA class
  \KOMAoptions{parskip=half}}
\makeatother
\usepackage{xcolor}
\IfFileExists{xurl.sty}{\usepackage{xurl}}{} % add URL line breaks if available
\IfFileExists{bookmark.sty}{\usepackage{bookmark}}{\usepackage{hyperref}}
\hypersetup{
  hidelinks,
  pdfcreator={LaTeX via pandoc}}
\urlstyle{same} % disable monospaced font for URLs
\usepackage[margin=1in]{geometry}
\usepackage{color}
\usepackage{fancyvrb}
\newcommand{\VerbBar}{|}
\newcommand{\VERB}{\Verb[commandchars=\\\{\}]}
\DefineVerbatimEnvironment{Highlighting}{Verbatim}{commandchars=\\\{\}}
% Add ',fontsize=\small' for more characters per line
\usepackage{framed}
\definecolor{shadecolor}{RGB}{248,248,248}
\newenvironment{Shaded}{\begin{snugshade}}{\end{snugshade}}
\newcommand{\AlertTok}[1]{\textcolor[rgb]{0.94,0.16,0.16}{#1}}
\newcommand{\AnnotationTok}[1]{\textcolor[rgb]{0.56,0.35,0.01}{\textbf{\textit{#1}}}}
\newcommand{\AttributeTok}[1]{\textcolor[rgb]{0.77,0.63,0.00}{#1}}
\newcommand{\BaseNTok}[1]{\textcolor[rgb]{0.00,0.00,0.81}{#1}}
\newcommand{\BuiltInTok}[1]{#1}
\newcommand{\CharTok}[1]{\textcolor[rgb]{0.31,0.60,0.02}{#1}}
\newcommand{\CommentTok}[1]{\textcolor[rgb]{0.56,0.35,0.01}{\textit{#1}}}
\newcommand{\CommentVarTok}[1]{\textcolor[rgb]{0.56,0.35,0.01}{\textbf{\textit{#1}}}}
\newcommand{\ConstantTok}[1]{\textcolor[rgb]{0.00,0.00,0.00}{#1}}
\newcommand{\ControlFlowTok}[1]{\textcolor[rgb]{0.13,0.29,0.53}{\textbf{#1}}}
\newcommand{\DataTypeTok}[1]{\textcolor[rgb]{0.13,0.29,0.53}{#1}}
\newcommand{\DecValTok}[1]{\textcolor[rgb]{0.00,0.00,0.81}{#1}}
\newcommand{\DocumentationTok}[1]{\textcolor[rgb]{0.56,0.35,0.01}{\textbf{\textit{#1}}}}
\newcommand{\ErrorTok}[1]{\textcolor[rgb]{0.64,0.00,0.00}{\textbf{#1}}}
\newcommand{\ExtensionTok}[1]{#1}
\newcommand{\FloatTok}[1]{\textcolor[rgb]{0.00,0.00,0.81}{#1}}
\newcommand{\FunctionTok}[1]{\textcolor[rgb]{0.00,0.00,0.00}{#1}}
\newcommand{\ImportTok}[1]{#1}
\newcommand{\InformationTok}[1]{\textcolor[rgb]{0.56,0.35,0.01}{\textbf{\textit{#1}}}}
\newcommand{\KeywordTok}[1]{\textcolor[rgb]{0.13,0.29,0.53}{\textbf{#1}}}
\newcommand{\NormalTok}[1]{#1}
\newcommand{\OperatorTok}[1]{\textcolor[rgb]{0.81,0.36,0.00}{\textbf{#1}}}
\newcommand{\OtherTok}[1]{\textcolor[rgb]{0.56,0.35,0.01}{#1}}
\newcommand{\PreprocessorTok}[1]{\textcolor[rgb]{0.56,0.35,0.01}{\textit{#1}}}
\newcommand{\RegionMarkerTok}[1]{#1}
\newcommand{\SpecialCharTok}[1]{\textcolor[rgb]{0.00,0.00,0.00}{#1}}
\newcommand{\SpecialStringTok}[1]{\textcolor[rgb]{0.31,0.60,0.02}{#1}}
\newcommand{\StringTok}[1]{\textcolor[rgb]{0.31,0.60,0.02}{#1}}
\newcommand{\VariableTok}[1]{\textcolor[rgb]{0.00,0.00,0.00}{#1}}
\newcommand{\VerbatimStringTok}[1]{\textcolor[rgb]{0.31,0.60,0.02}{#1}}
\newcommand{\WarningTok}[1]{\textcolor[rgb]{0.56,0.35,0.01}{\textbf{\textit{#1}}}}
\usepackage{graphicx,grffile}
\makeatletter
\def\maxwidth{\ifdim\Gin@nat@width>\linewidth\linewidth\else\Gin@nat@width\fi}
\def\maxheight{\ifdim\Gin@nat@height>\textheight\textheight\else\Gin@nat@height\fi}
\makeatother
% Scale images if necessary, so that they will not overflow the page
% margins by default, and it is still possible to overwrite the defaults
% using explicit options in \includegraphics[width, height, ...]{}
\setkeys{Gin}{width=\maxwidth,height=\maxheight,keepaspectratio}
% Set default figure placement to htbp
\makeatletter
\def\fps@figure{htbp}
\makeatother
\setlength{\emergencystretch}{3em} % prevent overfull lines
\providecommand{\tightlist}{%
  \setlength{\itemsep}{0pt}\setlength{\parskip}{0pt}}
\setcounter{secnumdepth}{-\maxdimen} % remove section numbering
\usepackage{tikz}
\usetikzlibrary{positioning,shapes.multipart,shapes}

\author{}
\date{\vspace{-2.5em}}

\begin{document}

\begin{center}\rule{0.5\linewidth}{0.5pt}\end{center}

title: ``Inference for categorical data'' author: ``Alexis Mekueko''
date: ``10/9/2020'' output: pdf\_document: default beamer\_presentation:
default html\_document: df\_print: paged slidy\_presentation: default
header-includes: -

\usepackage{tikz}

\begin{itemize}
\item
  \usetikzlibrary{positioning,shapes.multipart,shapes}
\end{itemize}

\begin{center}\rule{0.5\linewidth}{0.5pt}\end{center}

Github Link: \url{https://github.com/asmozo24/DATA606_Lab6}

\begin{Shaded}
\begin{Highlighting}[]
\KeywordTok{library}\NormalTok{(tidyverse) }\CommentTok{#loading all library needed for this assignment}
\KeywordTok{library}\NormalTok{(openintro)}
\KeywordTok{library}\NormalTok{(infer)}
\KeywordTok{library}\NormalTok{(gplots)}
\CommentTok{#head(fastfood)}
\CommentTok{#library(readxl)}
\CommentTok{#library(data.table)}
\CommentTok{#library(readr)}
\CommentTok{#library(plyr)}
\CommentTok{#library(dplyr)}
\CommentTok{#library(dice)}
\CommentTok{# #library(VennDiagram)}
\CommentTok{# #library(help = "dice")}
\CommentTok{#library(DBI)}
\CommentTok{#library(dbplyr)}

\CommentTok{#library(rstudioapi)}
\CommentTok{#library(RJDBC)}
\CommentTok{#library(odbc)}
\CommentTok{#library(RSQLite)}
\CommentTok{#library(rvest)}
\CommentTok{#library(stringr)}
\CommentTok{#library(readtext)}
\CommentTok{#library(ggpubr)}
\CommentTok{#library(fitdistrplus)}
\CommentTok{#library(ggplot2)}
\CommentTok{#library(moments)}
\CommentTok{#library(qualityTools)}
\KeywordTok{library}\NormalTok{(normalp)}
\CommentTok{#library(utils)}
\CommentTok{#library(MASS)}
\CommentTok{#library(qqplotr)}
\KeywordTok{library}\NormalTok{(DATA606)}
\end{Highlighting}
\end{Shaded}

\begin{verbatim}
## 
## Welcome to CUNY DATA606 Statistics and Probability for Data Analytics 
## This package is designed to support this course. The text book used 
## is OpenIntro Statistics, 3rd Edition. You can read this by typing 
## vignette('os3') or visit www.OpenIntro.org. 
##  
## The getLabs() function will return a list of the labs available. 
##  
## The demo(package='DATA606') will list the demos that are available.
\end{verbatim}

\begin{Shaded}
\begin{Highlighting}[]
\KeywordTok{getLabs}\NormalTok{()}
\end{Highlighting}
\end{Shaded}

\begin{verbatim}
##  [1] "Lab1"  "Lab2"  "Lab3"  "Lab4"  "Lab5a" "Lab5b" "Lab6"  "Lab7"  "Lab8" 
## [10] "Lab9"
\end{verbatim}

\begin{Shaded}
\begin{Highlighting}[]
\CommentTok{#library(StMoSim)}
\end{Highlighting}
\end{Shaded}

\hypertarget{the-data}{%
\subsection{The Data}\label{the-data}}

You will be analyzing the same dataset as in the previous lab, where you
delved into a sample from the Youth Risk Behavior Surveillance System
(YRBSS) survey, which uses data from high schoolers to help discover
health patterns. The dataset is called yrbss.

\hypertarget{exercise-1}{%
\subsubsection{Exercise 1}\label{exercise-1}}

What are the counts within each category for the amount of days these
students have texted while driving within the past 30 days?

\begin{Shaded}
\begin{Highlighting}[]
\KeywordTok{view}\NormalTok{(yrbss)}
\KeywordTok{str}\NormalTok{(yrbss)}
\end{Highlighting}
\end{Shaded}

\begin{verbatim}
## tibble [13,583 x 13] (S3: tbl_df/tbl/data.frame)
##  $ age                     : int [1:13583] 14 14 15 15 15 15 15 14 15 15 ...
##  $ gender                  : chr [1:13583] "female" "female" "female" "female" ...
##  $ grade                   : chr [1:13583] "9" "9" "9" "9" ...
##  $ hispanic                : chr [1:13583] "not" "not" "hispanic" "not" ...
##  $ race                    : chr [1:13583] "Black or African American" "Black or African American" "Native Hawaiian or Other Pacific Islander" "Black or African American" ...
##  $ height                  : num [1:13583] NA NA 1.73 1.6 1.5 1.57 1.65 1.88 1.75 1.37 ...
##  $ weight                  : num [1:13583] NA NA 84.4 55.8 46.7 ...
##  $ helmet_12m              : chr [1:13583] "never" "never" "never" "never" ...
##  $ text_while_driving_30d  : chr [1:13583] "0" NA "30" "0" ...
##  $ physically_active_7d    : int [1:13583] 4 2 7 0 2 1 4 4 5 0 ...
##  $ hours_tv_per_school_day : chr [1:13583] "5+" "5+" "5+" "2" ...
##  $ strength_training_7d    : int [1:13583] 0 0 0 0 1 0 2 0 3 0 ...
##  $ school_night_hours_sleep: chr [1:13583] "8" "6" "<5" "6" ...
\end{verbatim}

\begin{Shaded}
\begin{Highlighting}[]
\NormalTok{yrbss1 <-}\StringTok{ }\NormalTok{yrbss }\OperatorTok\StringTok{ }
\StringTok{  }\KeywordTok{count}\NormalTok{(text_while_driving_30d)}

\NormalTok{no_helmet <-}\StringTok{ }\NormalTok{yrbss }\OperatorTok
\StringTok{  }\KeywordTok{filter}\NormalTok{(helmet_12m }\OperatorTok{==}\StringTok{ "never"}\NormalTok{)}
\end{Highlighting}
\end{Shaded}

\end{document}
